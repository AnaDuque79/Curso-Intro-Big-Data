%%%%%%%%%%%%%%%%%%%%%%%%%%%%%%
\section{Almacenamiento de datos}
%%%%%%%%%%%%%%%%%%%%%%%%%%%%%%

\begin{frame}{Modelo de datos}
 \begin{wideitemize}
  \item Define el diseño y la implementación del sistema que almacena y gestiona
  los datos.
  \item Datos estructurados.
  \begin{itemize}
   \item Podemos almacenar sus valores en campos predefinidos con un tipo o
   una clase asociado de forma fija.
   \item Ejemplo: sistemas de bases de datos relacionales (RDBMS).
  \end{itemize}

 \end{wideitemize}

\end{frame}

%%---------------

\begin{frame}{Modelo de datos}
 \begin{wideitemize}
  \item Datos no estructurados
  \item No podemos predefinir de antemano su tipo, por lo que necesitamos un modelo
  de datos más flexible para su gestión.
  \item Relaciones complejas entre los diferentes elementos de datos.
  \item Ejemplo: Sistemas NoSQL (particionado por columnas, documentos, grafos,
  clave-valor, etc.).

 \end{wideitemize}

\end{frame}

%%---------------

\begin{frame}{Bases de datos relacionales}
 \begin{columns}[T]
    \begin{column}{.8\textwidth}
    \begin{wideitemize}
    \item Numerosas opciones en el mercado, propietarias o software libre.
    \item Larga trayectoria, tecnología muy madura y consolidada, permite predecir
    hasta cierto punto rendimientos esperados.
    \begin{itemize}
    \item Oracle, MySQL, MariaDB, PostgreSQL, SQLite, etc.
    \end{itemize}
    \item Gran variabilidad en cuanto a soporte para big data.
    \begin{itemize}
    \item Tipos de datos nativos.
    \item Particionado de tablas.
    \item Clustering, alta disponibilidad.
    \end{itemize}

    \item Object-Relational Mapping (ORM).
    \begin{itemize}
    \item Ejemplo: \href{http://www.sqlalchemy.org/}{SQLAlchemy} (Python).
    \end{itemize}

  \end{wideitemize}
    \end{column}
    \begin{column}{.2\textwidth}
    \vspace*{1.5cm}
    \includegraphics[width=0.8\textwidth]{figs/db.png}
    \end{column}
  \end{columns}

\end{frame}

%%---------------

\begin{frame}{Rendimiento RDBMS}
 \begin{columns}[T]
    \begin{column}{.8\textwidth}
    \begin{wideitemize}
    \item Es conveniente tener en cuenta algunas premisas importantes para
    analizar datos utilizando RDBMS.
    
    \item Utilizar motores \textit{ACID-compliant} únicamente cuando sea
    imprescindible.
    \begin{itemize}
     \item Si estamos trabajando con datos localmente, a los que solo accede uno o
     varios analistas, podemos usar otras opciones más rápidas.
     
     \item Ejemplos: MyISAM (MySQL), ARIA (MariaDB).
     
    \end{itemize}
    
  \item Desactivar claves primarias/foráneas en carga de datos. Después, utilizar
     solo cuando sea imprescindible.
     
  \item Activar particionado de tablas en diferentes ficheros (e.g. InnoDB).
  
  \item Cuidado con la codificación (asegurar UTF-8 siempre que sea posible).

  \end{wideitemize}
    \end{column}
    \begin{column}{.2\textwidth}
    \vspace*{1.5cm}
    \includegraphics[width=0.8\textwidth]{figs/db.png}
    \end{column}
  \end{columns}

\end{frame}

%%---------------

\begin{frame}{Rendimiento RDBMS}
 \begin{columns}[T]
    \begin{column}{.8\textwidth}
    \begin{wideitemize}
    \item La configuración por defecto de cualquier servidor de base de datos
    relacional no suele ser adecuada para el análisis de datos.
    
    \begin{itemize}
     \item Incrementar tamaño de los ficheros de claves (búsquedas, ordenación).
     
     \item Configurar adecuadamente el sistema de logging (errores, consultas
     lentas, etc.) ya que puede consumir rápidamente espacio en disco.
     
    \end{itemize}

   \item Formular las consultas basándonos siempre en conjuntos de datos y operaciones
   sobre los mismos (\texttt{JOIN}, \texttt{LEFT JOIN}, etc.).
   
   \begin{itemize}
    \item Evitar en lo posible cláusulas del tipo \texttt{WHERE X IN (...)}, ya
    que suelen involucrar búcles de búsqueda lentos.
   \end{itemize}

   \item Usar R y Python a partir de la fase de preparación
   y transformación.
   
  \end{wideitemize}
    \end{column}
    \begin{column}{.2\textwidth}
    \vspace*{1.5cm}
    \includegraphics[width=0.8\textwidth]{figs/db.png}
    \end{column}
  \end{columns}

\end{frame}

%%---------------

\begin{frame}{NoSQL}
 \begin{wideitemize}
  \item NoSQL = Not Only SQL.
  \item Escalabilidad y alto rendimiento para big data (en especial, tratamiento
  de tipos de datos muy heterogéneos o información textual).
  \item \textbf{Esquemas clave-valor}.
  \begin{itemize}
   \item Almacen de datos en pares clave-valor, no precisan esquema (Riak, Redis,
   Voldemort, etc.).
  \end{itemize}

  \item \textbf{Almacen de columnas}.
  \begin{itemize}
   \item Particionan datos por columnas, de forma que podemos paralelizar consultas
   sobre subconjuntos de datos muy grandes (HBase, Cassandra).
  \end{itemize}

 \end{wideitemize}

\end{frame}

%%---------------

\begin{frame}{Almacenamiento de datos: NoSQL}
 \begin{wideitemize}
  \item \textbf{Orientadas a documentos}.
  \begin{itemize}
   \item Cada clave está asociada a un documento, codificado según algún estándar
   de representación de datos (JSON, XML, YAML, etc.). 
   \item Los documentos pueden contener
   muchos pares clave-valor, clave-array (para listas de datos) u otros documentos.
   \item Ejemplo: MongoDB.
  \end{itemize}

  \item \textbf{Grafos}.
  \begin{itemize}
   \item Almacenan explícitamente información sobre nodos y sus relaciones, optimizando
   consultas que recorren los grafos (Neo4J).
   
   \item A cambio, tenemos que aprender un lenguaje de consultas muy diferente.
  \end{itemize}

 \end{wideitemize}

\end{frame}

%%---------------

\begin{frame}{Ventajas NoSQL}
 \begin{wideitemize}
  \item Mejor escalado horizontal, permite procesamiento paralelo.
  \begin{itemize}
   \item Consideran la agregación de nuevos recursos de computación ``en caliente''
   (autosharding, replicación).
  \end{itemize}

  \item No es necesario definir esquemas (tipos de datos), se pueden mezclar dinámicamente
  nuevos datos con los ya existentes (con limitaciones).
  
  \item Mejor integración con metodologías ágiles de desarrollo de software.
  \begin{itemize}
   \item Sprints cortos, prototipado rápido (vs. esquemas predefinidos).
   \item Dificultad para establecer a priori esquemas fijos de estructuras de datos.
  \end{itemize}

  \item Posibilidad de optimizar la recuperación de información mediante indexación
  (múltiple, dispersa, geoespacial, etc.).
  
 \end{wideitemize}

\end{frame}

%%---------------

\begin{frame}{Inconvenientes NoSQL}
 \begin{wideitemize}
  \item Requieren importantes conocimentos técnicos para su instalación, correcta
  configuración y administración (recordemos importancia del rendimiento).
  \item Todavía escasa madurez en comparación con RDBMS.
  \item Múltiples estándares de programación y APIs, incompatibles entre sí en
  muchos casos (necesidad de soporte de comunicación).
  \item Problema sistemas distribuidos: teorema CAP (\textit{consistencia}, \textit{disponibilidad},
  \textit{particionado}).
  
 \end{wideitemize}

\end{frame}

%%---------------

\begin{frame}{Ejemplo: Redis}
 \begin{wideitemize}
  \item Redis es un buen ejemplo de sistema de almacenamiento de datos NoSQL.
  
  \item Tremendamente popular, puesto que ofrece un buen rendimiento tanto con
  datos en memoria como con datos en disco.
  
  \item Buen enlace con Python: \textit{redis-py}
  \begin{itemize}
   \item \texttt{sudo pip install redis}
  \end{itemize}

  \item Ejemplo de programación con Python y Redis (Python notebook).
  
 \end{wideitemize}

\end{frame}

%%---------------

\begin{frame}{Sistemas de ficheros distribuidos}
 \begin{columns}[T]
    \begin{column}{.75\textwidth}
     \begin{wideitemize}
      \item Adecuados para distribuir datos sobre clústers de máquinas/clouds.
      \item Fuertemente ligados a tecnologías específicas.
      \item HDFS (Apache Hadoop).
      \item Google File System (GFS).
      \item Amazon S3.
    \end{wideitemize}
    \end{column}
    \begin{column}{.25\textwidth}
    \vspace*{1.5cm}
    \includegraphics[width=1.2\textwidth]{figs/cloud.png}
    \end{column}
  \end{columns}
 
\end{frame}
