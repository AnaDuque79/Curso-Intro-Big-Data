%% Big Data I: Ingeniería de datos
%
% Felipe Ortega, Javier M. Moguerza
% DEIO, URJC.
%
%
%%%%%%%%%%%%%%%%%%%%%%%%%%%%%%
\section{Big data: problemas y soluciones tecnológicas}
%%%%%%%%%%%%%%%%%%%%%%%%%%%%%%

%---------------

\begin{frame}{}
\begin{center}
 \huge Big data: problemas y soluciones tecnológicas
\end{center}
\end{frame}

%---------------

\begin{frame}{Mejor definición de big data hasta la fecha...}
\begin{center}
    \includegraphics[width=1\textwidth]{figs/Dan-Ariely-bigdata.jpeg}
\end{center}
\end{frame}

%---------------

\begin{frame}{Dimensiones big data}
\begin{wideitemize}
 \item Término controvertido, incluso para los propios profesionales.
 \item Consenso: definido por las 3 ``Vs'' [2-3].
 \begin{itemize}
  \item \textbf{Volumen} (tamaño, procesamiento).
  \item \textbf{Velocidad} (adquisición, procesamiento).
  \item \textbf{Variedad} (dimensiones).
 \end{itemize}
 \item A veces, se añade más factores (V's):
 \begin{itemize}
  \item \textit{veracidad} (integridad de datos, corrección...)
  \item \textit{valor} (el valor añadido que
 aporta big data para el negocio o dominio de aplicación).
  \item \textit{variabilidad}, \textit{visualización}, etc.
 \end{itemize}

 \item Lo importante: no es sólo una cuestión de tamaño.
\end{wideitemize}
 
\end{frame}

%%---------------

\begin{frame}{¿Cuántos son ``muchos datos''?}
\begin{wideitemize}
 \item Típicamente, más de los que podamos procesar en un sólo computador
 (incluso en un servidor muy potente).
 \begin{itemize}
  \item Por necesitar demasiada memoria.
  \item Por requerir demasiado espacio de almacenamiento.
  \item Porque no podemos almacenar el flujo de datos que nos llega de forma
  permanente (procesado \textit{streaming} vs. \textit{batch}).
  \item Porque necesitamos resultados con gran rapidez para tomar decisiones
  operativas.
 \end{itemize}
 
 \item A continuación, presentamos algunos ejemplos [4].

\end{wideitemize}
 
\end{frame}

%%---------------

\begin{frame}{Algunos números sobre big data}
 \begin{columns}[T]
    \begin{column}{.7\textwidth}
    \begin{wideitemize}
     \item \textbf{Walmart}.
     \begin{itemize}
      \item Fortune 500 Global.
      \item Mayor empleador privado del mundo (+2 millones empleados).
      \item Mayor distribuidor minorista del mundo.
     \end{itemize}

     \item Sus servidores procesan más de un millón de transacciones de clientes
     cada hora.
     \item Sus bases de datos almacenan más de 2,5 Petabytes (1 Petabyte = 1024 Terabytes).
     
    \end{wideitemize}

    \end{column}
    \begin{column}{.3\textwidth}
    \vspace*{1cm}
    \hspace*{-0.5cm}
    \includegraphics[width=1\textwidth]{figs/Walmart-exterior.jpg}
    \end{column}
  \end{columns}

\end{frame}

%%---------------

\begin{frame}{Algunos números sobre big data}
 \begin{columns}[T]
    \begin{column}{.7\textwidth}
    \begin{wideitemize}
     \item \textbf{LHC (CERN)}.
     \begin{itemize}
      \item Mayor y más potente colisionador de partículas del mundo.
      \item Una de las mayores fuentes de datos de experimentos científicos del
      mundo.
     \end{itemize}

     \item Se estima que genera unos 15 Petabytes de información anualmente.
     \item Se analizan en un sistema computacional distribuído y tolerante a fallos
     (grid computing):
     \begin{itemize}
      \item 170 centros de computación,
      \item 36 países participantes.
      \item Red global de comunicación.
     \end{itemize}

    \end{wideitemize}

    \end{column}
    \begin{column}{.3\textwidth}
    \vspace*{2cm}
    \hspace*{-0.5cm}
    \includegraphics[width=1\textwidth]{figs/CERN-LHC.jpg}
    \end{column}
  \end{columns}

\end{frame}

%%---------------

\begin{frame}{Algunos números sobre big data}
 \begin{columns}[T]
    \begin{column}{.7\textwidth}
    \begin{wideitemize}
     \item \textbf{Datos en la Web}.
     \item \textbf{Facebook} opera sobre 500 Terabytes de información de registro de actividad
     de sus usuarios, y sobre cientos de Terabytes de imágenes.
     \item Cada minuto se cargan 100 horas de vídeo en \textbf{Youtube}, y más de 135.000
     horas de vídeo son vistas.
     \item \textbf{Twitter} sirve a casi 600 millones de usuarios que generan 9.100 tweets
     cada segundo.
     \item Los sistemas de \textbf{eBay} procesan más de 100 Petabytes de información al día.

    \end{wideitemize}

    \end{column}
    \begin{column}{.3\textwidth}
    \vspace*{2cm}
    \hspace*{-0.5cm}
    \includegraphics[width=1\textwidth]{figs/Internet-map.png}
    \end{column}
  \end{columns}

\end{frame}

%%---------------

\begin{frame}{Algunos números sobre big data}
 \begin{columns}[T]
    \begin{column}{.7\textwidth}
    \begin{wideitemize}
     \item \textbf{Sector aeronáutico}.
     \item Un avión comercial de Boeing puede generar alrededor de 10 Terabytes
     de información operacional cada 30 minutos de funcionamiento.
     \item Por tanto, en un vuelo transatlántico se pueden llegar a generar
     varios cientos de Terabytes de información.
     \item Se realizan alrededor de 22.000 vuelos diarios en todo el mundo.
     \item Esto nos ofrece una idea de la ingente cantidad de datos generada
     por máquinas y redes de sensores de manera regular.

    \end{wideitemize}

    \end{column}
    \begin{column}{.3\textwidth}
    \vspace*{2cm}
    \hspace*{-0.5cm}
    \includegraphics[width=1\textwidth]{figs/Emirates-Landing.jpg}
    \end{column}
  \end{columns}

\end{frame}

%%---------------

\begin{frame}{Necesidades computacionales}
 \begin{columns}[T]
    \begin{column}{.7\textwidth}
    \begin{wideitemize}
     \item Precisamos potencia y capacidad de computación para ingeniería de
     datos.
     \item Problema: el tráfico de datos crece a mayor velocidad que nuestra
     capacidad de computación.
     \begin{itemize}
      \item (2002-2009): volúmen global del tráfico de datos se 
      multiplicó por 56; potencia de computación se multiplicó sólo por 16.
      \item (1998-2005): centros de datos crecieron en tamaño un 173\%
      anual [4], mientras que la eficiencia en consumo energético no mejoró a la par.
      \item Esto generará una enorme \textit{huella de consumo energético} para análisis
      de datos.
      \item 50\% de los centros de cómputo de datos (aprox.) solo funcionan
      al 50\% de su rendimiento máximo.
     \end{itemize}


    \end{wideitemize}

    \end{column}
    \begin{column}{.3\textwidth}
    \vspace*{1.7cm}
    \includegraphics[width=0.65\textwidth]{figs/jcartier-Cluster.png}
    \end{column}
  \end{columns}

\end{frame}

%%---------------

\begin{frame}{Tipos de datos según su estructura}

\begin{wideitemize}
  \item \textbf{Datos estructurados}: Tienen una serie de campos con significado
   predefinido. Cada campo está asociado a un tipo de datos (numérico, textual,
   doble precisión, objeto serializado...). Ejemplo: RDBMS.
   
   \item \textbf{Datos semi-estructurados}: Se representan mediante un formato
   de codificación que aporta cierta estructura e información sobre los datos
   (metadatos). Sin embargo, su contenido (número de campos, formato de cada
   campo, etc.) puede ser muy variado. Ej: documentos XML.
   
   \item \textbf{Datos no estructurados}: El formato de los datos no está claramente
   definido de forma previa. Pueden aparecer mezclados datos numéricos, textuales
   o multimedia, y en un orden imprevisible.

\end{wideitemize}

\end{frame}

%%---------------

\begin{frame}{Soluciones para datos no estructurados}
    \begin{wideitemize}
     \item Necesitamos tecnologías y métodos flexibles para gestionar este tipo de
     fuentes de datos (necesidades dinámicas e imprevisibles). Ejemplo:
     \textbf{tecnologías NoSQL}.
     \begin{itemize}
      \item Ejemplo: Esquemas \textbf{clave-valor}.
      
      \item Almacenan duplas (\textit{clave-valor}), donde las claves asociadas a
      cada valor son únicas (para acelerar las búsquedas) y los valores pueden ser
      también objetos complejos (tales como listas, tablas hash, etc).
      
      \item Ejemplo: Bases de datos \textbf{documentales}.
      
      \item Almacenan documentos representados en cierto formato de condificación
      (tales como XML, JSON o YAML).
      
      \item También siguen un esquema de almacenamiento \textit{clave-valor}, pero
      el contenido de los documentos es arbitrario, y además se ofrecen mecanismos para
      realizar búsquedas basadas en dichos contenidos (utilizando los metadatos del
      sistema de condificación).
     \end{itemize}

    \end{wideitemize}

\end{frame}

%%---------------

\begin{frame}{Tipos de procesamiento de datos}
 \begin{wideitemize}
 
 \item Clasificación de procesamiento de datos según requisitos de interacción:
 
\end{wideitemize}

\begin{enumerate}
   \item Procesamiento \textbf{batch} (también llamado \textit{offline}): No
   existen requisitos estrictos en cuanto al tiempo que podemos emplear en
   la preparación, transformación y computación de los datos almacenados.
   Ejemplo: MapReduce (Hadoop).\\
   
   \item Procesamiento de \textbf{flujos de datos} (\textit{streaming}, también
   llamado \textit{online}): Existen requisitos estrictos sobre el tiempo máximo
   que podemos emplear para preparar, trasformar y procesar los datos. Puede
   deberse a varias razones:
   \begin{itemize}
    \item Análisis interactivo.
    \item Interacción con usuarios finales (servicios, dashboards, etc.).
    \item Excesiva velocidad o volumen de datos (no podemos almacenar localmente).
   \end{itemize}


  \end{enumerate}

\end{frame}

%%---------------

\begin{frame}{Tipos de procesamiento de datos}
 \begin{columns}[T]
    \begin{column}{.5\textwidth}
    \begin{figure}
    \includegraphics[width=0.65\textwidth]{figs/conveyor-sushi.jpg} 
    \caption{Procesamiento \textit{streaming}}
    \end{figure}

    \end{column}
    \begin{column}{.5\textwidth}
    \vspace*{1.7cm}
    \begin{figure}
    \includegraphics[width=0.9\textwidth]{figs/spices-store.jpg}
    \caption{Procesamiento \textit{batch}}
    \end{figure}
    \end{column}
  \end{columns}

\end{frame}

%%---------------

\begin{frame}{Esquema procesamiento \textit{batch}}
 \includegraphics[width=\textwidth]{figs/procesamiento-batch.png}
\end{frame}

%%---------------

\begin{frame}{Esquema procesamiento \textit{streaming}}
 \includegraphics[width=0.95\textwidth]{figs/procesamiento-streaming.png}
\end{frame}

%%---------------

\begin{frame}{Tendencias procesamiento de datos}
 \begin{wideitemize}
 
 \item El procesamiento \textbf{streaming se está imponiendo} rápidamente.
 
 \item MapReduce no es suficientemente flexible ni rápido para muchos problemas
 de análisis de datos.
 
 \begin{itemize}
  \item \href{http://www.datacenterknowledge.com/archives/2014/06/25/google-dumps-mapreduce-favor-new-hyper-scale-analytics-system/}
  {Junio 2014: Google declara que dejaron de usar MapReduce hace años.}
 \end{itemize}

 \item MapReduce no es adecuado para muchos modelos de análisis de datos, que
 incluyen operaciones iterativas.
 
 \begin{itemize}
  \item Exigen un importante esfuerzo para programar estos procesos de modo que
  se reduzca el número de pasadas sobre los datos (cada iteración es muy cara).
 \end{itemize}

 \item Por contra, los sistemas de procesado \textit{streaming} se pueden adaptar
 a muchos más tipos de análisis, han sido concecibos para ser rápidos y escalables.
 
\end{wideitemize}

\end{frame}

%%---------------

\begin{frame}{Tendencias procesamiento de datos}
 
 \begin{wideitemize}
  
  \item En procesado \textit{streaming} se crean flujos de \textbf{datos inmutables}, que
  se procesan o transforman para generar nuevos flujos de datos. Se puede añadir 
  cierta \textit{persistencia}.
  
  \item También es posible combinar \textit{streaming} con procesado \textit{batch},
  (la llamada \textbf{arquitectura lambda}) pero cuidado con la duplicidad de trabajo.
  
  \item Pero exige ciertos requisitos adicionales:
  
  \begin{itemize}
   \item Sistemas de colas de mensajes / buffer de entrada que almacenen temporalmente
   los datos hasta que entren al flujo de procesado (idealmente sin péridas).
   
   \item Incluir sistemas automáticos de distribución de carga y tolerancia ante
   fallos de nodos de procesamiento.
   
  \end{itemize}

  
 \end{wideitemize}

 
\end{frame}


%%---------------

\begin{frame}{Bibliografía}
\begin{enumerate}
 \item Provost, F., Fawcett, T. Data Science for Business. O'Reilly Media Inc. Julio 2013.
 \item Cathy O'Neil, Rachel Schutt. Doing Data Science: Straight Talk from the Frontline.
 O'Reilly Media Inc. Octubre 2013.
 \item Doug Laney. 3d Data management: controlling data volume, velocity and variety.
 Appl. Delivery Strategies Meta Group (949)(2001).
 \item Kambatla, K. et al. Trends in big data analytics. Journal of Parallel and Distributed
 Computing (in press). Elsevier. Enero 2014.
\end{enumerate}
\end{frame}

%%---------------

\begin{frame}{Créditos}
\begin{enumerate}
 \item Imagen Walmart-exterior.jpg por see. CC-BY-SA-3.0, via Wikimedia Commons.
 \item Imagen inside-CERN-LHC por Juhanson. CC-BY-SA-3.0, via Wikimedia Commons.
 \item Imagen Internet map por The Opte Project. CC-BY-2.5 , via Wikimedia Commons.
 \item Imagen Boeing Emirates por Faisal Akram desde Dhaka, Bangladesh. CC-BY-SA-2.0, via Wikimedia Commons
 \item Imágenes clipart obtenidas de Openclipart, todas ellas disponibles en dominio público.
 \item Todos los logos de proyectos y/o empresas son marcas registradas, utilizados simplemente con fines ilustrativos.
\end{enumerate}
\end{frame}

%%--------------

\begin{frame}{Contacto}
\begin{huge}
e-mail: felipe.ortega@urjc.es\\~\\
Twitter: @jfelipe
\end{huge}
\end{frame}

%%---------------
