%%%%%%%%%%%%%%%%%%%%%%%%%%%%%%
\section{Obtención de datos}
%%%%%%%%%%%%%%%%%%%%%%%%%%%%%%

\begin{frame}{}
\begin{center}
 \huge Obtención de datos
\end{center}
\end{frame}

%%---------------

\begin{frame}{Obtención de datos}
 \begin{columns}[T]
    \begin{column}{.7\textwidth}
  \begin{wideitemize}
  \item Etapa crucial y con frecuencia infravalorada.
  \item Con frecuencia, la obtención y preparación de datos consume cerca del
  \textbf{85\% del tiempo total} del proyecto de análisis de datos.

  \item Diferentes retos.
  \begin{itemize}
   \item Multiplicidad de fuentes.
   \item Métodos de obtención de datos (\textit{scrapping}, \textit{streaming}, APIs...).
   \item Diferentes formatos de representación.
   \item Consolidación de datos obtenidos.
  \end{itemize}

 \end{wideitemize}
    \end{column}
    \begin{column}{.30\textwidth}
    \vspace*{0.8cm}
    \includegraphics[width=1\textwidth]{figs/input.png}
    \end{column}
  \end{columns}

\end{frame}

%%---------------

\begin{frame}{Obtención de datos}
 \begin{wideitemize}
  \item Multiplicidad de fuentes
\end{wideitemize}
  \begin{figure}
   \centering
   \includegraphics[width=.95\textwidth]{figs/multisource.png}
  \end{figure}

\end{frame}
\begin{frame}{Obtención de datos: aspectos de diseño}
 \begin{wideitemize}
  \item Construir módulos intercambiables para manejar cada tipo de fuente.
  \item Misma interfaz de uso, ocultando peculiaridades del manejo de cada
  fuente o tipo de datos.
  \item Considerar diseños basados en colas de elementos (datos o bloques de
  datos de entrada) que permitan gestionar:
  \begin{itemize}
   \item Distintas velocidades de adquisición.
   \item Datos heterogéneos.
   \item Mantenimiento estricto del orden de llegada.
  \end{itemize}

 \end{wideitemize}

\end{frame}

%%---------------

\begin{frame}{Obtención de datos: aspectos de diseño}
 \begin{wideitemize}
  \item Nunca debemos asumir que las fuentes nos van a enviar los datos
  correctamente representados o íntegros.
  \item Ejemplo: Beautiful Soup.
  \begin{itemize}
   \item Biblioteca Python para adquisición de datos HTML (y XML).
   \item Soporta fallos en sintaxis HTML (o XML) de los documentos de origen.
  \end{itemize}

 \end{wideitemize}

\end{frame}

%%---------------

\begin{frame}{Obtención de datos: aspectos de diseño}
 \begin{columns}[T]
    \begin{column}{.8\textwidth}
  \begin{wideitemize}
  \item Aquí importa (y mucho) la velocidad de ejecución.
  \begin{itemize}
   \item En flujos de datos en tiempo real podemos perder datos si no los recuperamos
   a tiempo.
   \item Los tiempos de espera para tratamiento de fuentes de gran volumen se pueden 
   alargar demasiado (días, semanas).
  \end{itemize}

  \item Ejemplos: lxml, UJSON (Python).
  
 \end{wideitemize}
    \end{column}
    \begin{column}{.2\textwidth}
    \vspace*{0.8cm}
    \includegraphics[width=0.8\textwidth]{figs/cronometer.png}
    \end{column}
  \end{columns}

\end{frame}

%%---------------

\begin{frame}{Obtención de datos: aspectos de diseño}
 \begin{columns}[T]
    \begin{column}{.8\textwidth}
  \begin{wideitemize}
  \item Pero también hay que respetar los límites impuestos por determiandos sistemas
  fuente.
  \begin{itemize}
   \item En APIs públicas, se suele limitar el número de consultas que pueden realizarse
   en un cierto intervalo, también la cantidad de datos devueltos por cada consulta o
   el rango temporal que podemos abarcar.
  \end{itemize}

  \item Ejemplos: \href{https://dev.twitter.com/docs/rate-limiting/1.1}{Twitter REST API 1.1}, 
  \href{https://developers.facebook.com/docs/reference/ads-api/api-rate-limiting/}{Facebook}.
  
 \end{wideitemize}
    \end{column}
    \begin{column}{.2\textwidth}
    \vspace*{0.8cm}
    \includegraphics[width=0.8\textwidth]{figs/cronometer.png}
    \end{column}
  \end{columns}

\end{frame}

%%---------------

%%%%%%%%%%%%%%%%%%%%%%%%%%%%%%
\section{Representación de datos}
%%%%%%%%%%%%%%%%%%%%%%%%%%%%%%

\begin{frame}{Representación de datos}
 \begin{columns}[T]
    \begin{column}{.75\textwidth}
  \begin{wideitemize}
  \item Formatos relacionados con tecnologías web.
  \begin{itemize}
   \item HTML, XML, JSON, YAML, etc.
  \end{itemize}

  \item Procesamiento.
  \begin{itemize}
   \item CSV, HDF5, ff, otros formatos específicos.
  \end{itemize}
  
  \item Metadatos.
  \begin{itemize}
   \item RDF (datos enlazados).
  \end{itemize}

 \end{wideitemize}
    \end{column}
    \begin{column}{.25\textwidth}
    \vspace*{0.8cm}
    \includegraphics[width=0.95\textwidth]{figs/binarydoc.png}
    \end{column}
  \end{columns}

\end{frame}

%%---------------

\begin{frame}{Representación de datos}
\begin{itemize}
 \item Ejemplos [3]: JSON
\end{itemize}

 \begin{figure}
  \centering
  \includegraphics[width=0.5\textwidth]{figs/json-sample.jpeg} 
\end{figure}

\end{frame}

%%---------------

\begin{frame}{Representación de datos}
\begin{itemize}
 \item Ejemplos [3]: YAML
\end{itemize}

 \begin{figure}
  \centering
  \includegraphics[width=0.5\textwidth]{figs/yaml-sample.jpeg} 
\end{figure}

\end{frame}

%%---------------

\begin{frame}{Representación de datos}
\begin{itemize}
 \item Ejemplos [3]: XML
\end{itemize}

 \begin{figure}
  \centering
  \includegraphics[width=1\textwidth]{figs/xml-sample.jpeg} 
\end{figure}

\end{frame}

%%---------------

\begin{frame}{Representación de datos}
\begin{wideitemize}
 \item Benchmark bibliotecas serialización (Python) [4].
\end{wideitemize}

 \begin{figure}
  \centering
  \includegraphics[width=1\textwidth]{figs/benchmark-serial.pdf} 
\end{figure}

\end{frame}

%%---------------

\begin{frame}{Representación de datos: procesamiento}
 \begin{wideitemize}
  \item Almacenamiento de estructuras de datos de gran tamaño en disco.
  
  \item Estándares
  \begin{itemize}
   \item Hyerarchical Data Format version 5 (HDF5).
  \end{itemize}
  
  \item Otros formatos específicos.
  \begin{itemize}
   \item Paquetes R \texttt{ff}, \texttt{ffbase} o \texttt{bigmemory}.
  \end{itemize}


 \end{wideitemize}

\end{frame}

%%---------------

\begin{frame}{Representación de datos: HDF5}
 \begin{wideitemize}
  \item Conjunto de datos jerárquicos, estructurados y autodescriptivos (metadatos).
  \item Capaz de escalar con facilidad al nivel de Exabyte (\textasciitilde1000 TB), compresión
  transparente, ubicación en múltiples dispositivos.
  \item Capacidad de indexación y E/S parcial.
  \begin{itemize}
   \item Evitamos cargar grandes volúmenes de datos en memoria o búsquedas secuenciales.
  \end{itemize}
  
  \item Bibliotecas disponibles en C, C++, Python, MATLAB, etc. 

 \end{wideitemize}

\end{frame}

%%---------------

\begin{frame}{Representación de datos: HDF5}
 \begin{wideitemize}
  \item Recomendable cuando los datos sean [5]:
  \begin{itemize}
   \item Grandes arrays numéricos.
   \item De tipo homogéneo.
   \item Que se puedan organizar jerárquicamente.
   \item Con metadatos de tipo arbitrario.
  \end{itemize}
  
  \item Para gestión de relaciones entre datos mejor usar bases de datos.
  \item Se puede usar también formatos más sencillos (e.g. CSV) para casos
  simples.

 \end{wideitemize}

\end{frame}

%%---------------

\begin{frame}{Representación de datos: otros formatos}
 \begin{wideitemize}
  \item Proyecto \texttt{ff} para el lenguaje R.
  \item Permite manejar grandes volúmenes de datos en R, sin necesidad de recurrir
  a clusters o cloud computing.
  \item Implementación de estructuras de datos comunes en R (ej. data frames).
  \item Implementación en C y C++ a bajo nivel, transparente para el usuario.
  \item Soporte para aplicación paralela de operaciones sobre datos en disco.
 \end{wideitemize}

\end{frame}

%%---------------

\begin{frame}{Representación de datos: RDF}
 \begin{wideitemize}
  \item Resource Description Framework.
  \item Familia de estándares de representación de metadatos promovida por W3C.
  \item Tripletas (sujeto-predicado-objeto) definen grafos dirigidos.
  \item Ofrecen información sobre ubicación y relaciones entre los datos almacenados
  (recursos web enlazados).
  \item Es posible consultar el grafo mediante el lenguaje \texttt{SPARQL}.
 \end{wideitemize}

\end{frame}

%%---------------

\begin{frame}{Representación de datos}
\begin{wideitemize}
 \item Ejemplo grafo RDF.
\end{wideitemize}

 \begin{figure}
  \centering
  \includegraphics[width=0.7\textwidth]{figs/Rdf-graph-Eric-Miller.png} 
\end{figure}

\end{frame}

%%---------------
